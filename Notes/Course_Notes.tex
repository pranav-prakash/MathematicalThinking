\documentclass[]{article}
\usepackage{enumerate}
\usepackage{amsmath}
\usepackage{framed}

\begin{document}

\title{Notes for Introduction to Mathematical Thinking}
\author{Pranav Prakash \cr Rohit Sriram}
\maketitle

\section{Week 1}

Introduction to symbols:

\begin{center}

\begin{framed}
	$\wedge$ means a logical conjunction \\
 	$\vee$ means a logical disjunction \\
 	$\neg$ is negation
\end{framed}

Truth Tables for these operators \\
\bigskip
\bigskip
Conjunction \\
\bigskip

\begin{tabular}{c c | c}
$\phi$ & $\psi$ & $\phi \wedge \psi$ \\ \hline
T & T & T \\
T & F & F \\
F & T & F \\
F & F & F
\end{tabular} \\

\bigskip
\bigskip
Disjunction \\
\bigskip

\begin{tabular}{c c | c}
$\phi$ & $\psi$ & $\phi \vee \psi$ \\ \hline
T & T & T \\
T & F & T \\
F & T & T \\
F & F & F
\end{tabular}

\bigskip
\bigskip
Negation \\
\bigskip

\begin{tabular}{c | c}
$\phi$ & $\neg \phi$ \\ \hline
T & F \\
F & T
\end{tabular}


\end{center}

\section{Week 2}

\section{Week 3}
	\section{Lecture 3: Implication}
	
		\begin{center}
			
				Implication has two parts, a truth part, and a causation part.\\ 
				For the purpose of this course we are ignoring the causation part.\\
				\begin{framed}
					The truth part is called the conditional or sometimes the material conditional.\\
								$\Rightarrow$ means implication.\\
				\end{framed}
			
				In a conditional statement you have a statement $\phi$ $\Rightarrow$ $\psi$.\\
				In this statement $\phi$ is the antecedent, and $\psi$ is the consequent.\\
				\begin{framed}
				The truth or falsity of the statement $\phi$ $\Rightarrow$ $\psi$ depends upon the truth values of $\phi$ and $\psi$.\\
				\end{framed}
				
				This means that you can use truth tables to find the truth or falsity of an implication statement.
				\begin{framed}
				For any implication statement, if $ \phi $ is true and $ \psi $ is false, then the implication statement is false. For any other case, the implication statement is true.
				\end{framed}
				
				
		
		\end{center}

\end{document}
